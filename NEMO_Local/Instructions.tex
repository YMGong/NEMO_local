%%%%%%%%%%%%%%%%%%%%%%%%%%%%%%%%%%%%%%%%%
% Code Snippet
% LaTeX Template
% Version 1.0 (14/2/13)
%
% This template has been downloaded from:
% http://www.LaTeXTemplates.com
%
% Original author:
% Velimir Gayevskiy (vel@latextemplates.com)
%
% License:
% CC BY-NC-SA 3.0 (http://creativecommons.org/licenses/by-nc-sa/3.0/)
%
%%%%%%%%%%%%%%%%%%%%%%%%%%%%%%%%%%%%%%%%%

\documentclass[11pt]{article}

%----------------------------------------------------------------------------------------

\usepackage{listings} % Required for inserting code snippets
\usepackage[usenames,dvipsnames]{color} % Required for specifying custom colors and referring to colors by name
\usepackage{blindtext} % for dummy text

\usepackage{hyperref}
\hypersetup{
    colorlinks=true,
    linkcolor=blue,
    filecolor=magenta,      
    urlcolor=cyan,
}

\definecolor{DarkGreen}{rgb}{0.0,0.4,0.0} % Comment color
\definecolor{Grey}{rgb}{0.6,0.6,0.6} % Comment color
\definecolor{highlight}{RGB}{255,251,204} % Code highlight color


\lstdefinestyle{Style1}{ % Define a style for your code snippet, multiple definitions can be made if, for example, you wish to insert multiple code snippets using different programming languages into one document
language=Perl, % Detects keywords, comments, strings, functions, etc for the language specified
backgroundcolor=\color{highlight}, % Set the background color for the snippet - useful for highlighting
basicstyle=\footnotesize\ttfamily, % The default font size and style of the code
breakatwhitespace=false, % If true, only allows line breaks at white space
breaklines=true, % Automatic line breaking (prevents code from protruding outside the box)
captionpos=b, % Sets the caption position: b for bottom; t for top
commentstyle=\usefont{T1}{pcr}{m}{sl}\color{DarkGreen}, % Style of comments within the code - dark green courier font
deletekeywords={}, % If you want to delete any keywords from the current language separate them by commas
%escapeinside={\%}, % This allows you to escape to LaTeX using the character in the bracket
firstnumber=1, % Line numbers begin at line 1
frame=single, % Frame around the code box, value can be: none, leftline, topline, bottomline, lines, single, shadowbox
frameround=tttt, % Rounds the corners of the frame for the top left, top right, bottom left and bottom right positions
keywordstyle=\color{Blue}\bf, % Functions are bold and blue
morekeywords={}, % Add any functions no included by default here separated by commas
numbers=left, % Location of line numbers, can take the values of: none, left, right
numbersep=10pt, % Distance of line numbers from the code box
numberstyle=\tiny\color{Gray}, % Style used for line numbers
rulecolor=\color{black}, % Frame border color
showstringspaces=false, % Don't put marks in string spaces
showtabs=false, % Display tabs in the code as lines
stepnumber=5, % The step distance between line numbers, i.e. how often will lines be numbered
stringstyle=\color{Purple}, % Strings are purple
tabsize=2, % Number of spaces per tab in the code
}

% Create a command to cleanly insert a snippet with the style above anywhere in the document
\newcommand{\insertcode}[2]{\begin{itemize}\item[]\lstinputlisting[caption=#2,label=#1,style=Style1]{#1}\end{itemize}} % The first argument is the script location/filename and the second is a caption for the listing

%----------------------------------------------------------------------------------------
\begin{document}
\begin{titlepage}
   \vspace*{\stretch{1.0}}
   \begin{center}
      \Large\textbf{Instructions for using NEMO on SISU}\\[4\baselineskip]
      \large\textit{Yongmei Gong}\\
      \large\textit{12 April 2018}
   \end{center}
   \vspace*{\stretch{2.0}}
\end{titlepage}

The following instruction consist text instructions and a number of well commented bash files on how to launch basic nemo simulations. They are created for the environment of the super computer Sisu in CSC-IT center for science.\\
\tableofcontents
\newpage
\section{Build and use NEMO on Sisu}
\subsection{Get the code}
1. Sign up for the wiki (http://forge.ipsl.jussieu.fr/nemo/wiki/Users) by sending an email containing your username (5 characters minimum length) to nemo@forge.ipsl.jussieu.fr;\\
2. Get the conformation email then reset your password.\\
3. Now download the source code from the distribution
%----------------------------------------------------------------------------------------
\insertcode{"./get_nemo_code.bash"}{} % The first argument is the script location/filename and the second is a caption for the listing
%----------------------------------------------------------------------------------------
\subsection{Declare the compilers}
All compiler options in NEMO are controlled using files in NEMOGCM/ARCH/arch.\\
Now we create a file to declare the compilers we use to build nemo accroding to what we have in Sisu.
%----------------------------------------------------------------------------------------
\insertcode{"./creat_compiler_links.bash"}{} % The first argument is the script location/filename and the second is a caption for the listing
%----------------------------------------------------------------------------------------
\subsection{Build NEMO for experiment - GYRE\_XIOS}
Now we build an executable for the experiment GYRE\_XIOS.\\
We use the up-to-date version of XIOS - xios\/2.0.990 instead the oldder xios\/1.0. This requires declaring the active cpp keys in the cpp\_*\_.fcm files.
%----------------------------------------------------------------------------------------
\insertcode{"./launch_GYRE_XIOS.bash"}{} % The first argument is the script location/filename and the second is a caption for the listing
%----------------------------------------------------------------------------------------
\subsection{Build NEMO for experiment - ORCA2\_LIM3}
Now we build an executable for the experiment ORCA2\_LIM3.\\
ORCA is the generic name given to global ocean configurations. Its specificity lies on the horizontal curvilinear mesh used to overcome the North Pole singularity found for geographical meshes. LIM (Louvain-la-Neuve sea-ice model, multi-category model LIM3 is used) is a thermodynamic-dynamic sea ice model specifically designed for climate studies.\\
Similarly this requires declaring the active cpp keys in the cpp\_*\_.fcm files. And this time the experiment needs input data.
%----------------------------------------------------------------------------------------
\insertcode{"./launch_ORCA2_LIM3.bash"}{} % The first argument is the script location/filename and the second is a caption for the listing
%----------------------------------------------------------------------------------------
\subsection{Build NEMO for experiment - ORCA1\_LIM3}
The difference between ORCA1 and ORCA2 is that the spacial resolution of the former is 1 degree and the latter is 2 degree. So ORCA1 has higher resolution and also needs more input data.\\
You will need files and data in two folders: ORCA1\_cfg and ORCA1\_data to create ORCA1 experiment configuration
%----------------------------------------------------------------------------------------
\insertcode{"./launch_ORCA1_LIM3.bash"}{} % The first argument is the script location/filename and the second is a caption for the listing
%----------------------------------------------------------------------------------------
\subsection{Use nemo for long transient runs}
Note below is just a template file. To be able to run it you need additional files which you can get by: \\
\colorbox{Grey}{\texttt{>>}git clone https://version.helsinki.fi/hydro\_geophysics/nemo\_scripts.git} \\
However you need to get the permission to access. \\
Then you need to ask data from Petteri.\\
\insertcode{"../NEMO_sh_script/N101.sh"}{} 
\subsection{Clean nemo build}
If something has gone wrong with or has been changed for the build of nemo (nemo.exe)\\
-	Clean a bad configuration \\
\colorbox{Grey}{\texttt{>>}./makenemo -C YOUR\_CONFIG clean\_config}\\
-	Uninstalling (Clean up the whole thing)\\
\colorbox{Grey}{\texttt{>>}./makenemo -n YOUR\_BUILD clean}\\
(e.g. \colorbox{Grey}{\texttt{>>}./makenemo ./makenemo -t \$TMPDIR -m XC40-SISU -n MY\_GYRE\_XIOS clean})
\subsection{Check the run status and outputs}
- If run crushes, try to find 'E R R O R' section in ocean.output 
- Use ncview (now it is only in Petteri’s directory) to check the results *nc files
\colorbox{Grey}{\texttt{>>}cd /homeappl/home/puotila/bin}\\
\colorbox{Grey}{\texttt{>>}export HDF5\_DISABLE\_VERSION\_CHECK=1}\\
\colorbox{Grey}{\texttt{>>}./ncview /YOUR\_OUTPUT\_DIR/\*\_5d\_00010101\_00011230\_grid\_T.nc \&}\\
\subsection{Build nemo tools}
TBA
\section{Build and use barakuda on Taito}
It is the best to build barakuda on taito
\subsection{Get the code}
Get the version from Petteri's git (you need permission)\\
\colorbox{Grey}{\texttt{>>} git clone https://version.helsinki.fi/pjuotila/barakuda.git}\\
\subsection{build up basemap}
Before you get barakuda running you need to add lots of module packages for python.\\
\insertcode{"../barakuda_script/basemap_install.sh"}{} 
\subsection{build up cdftools}
Go to folder cdftools\_light. \\
1. make sure you have these modules. Otherwise load them. If can't make sure you login to taito-login4.\\
 \colorbox{Grey}{\texttt{>>} module load intel python-env/intelpython3.5 hdf5-par netcdf4}\\
2. link the make file.\\
 \colorbox{Grey}{\texttt{>>} ln -s  make.macro  macro/macro.ifort\_taito}\\
3. Make sure you use intel and then make \\
 \colorbox{Grey}{\texttt{>>} gmake}\\
\subsection{Use barakuda}
1. Create your configuration file.\\
 For example use configs/config\_ORCA1\_L75\_taito.sh \\
 Change NEMO\_OUT\_STRCT, DIAG\_DIR, CONF\_INI\_DIR accordingly \\
2. Get data from Petteri and put them in CONF\_INI\_DIR\\
3. Tell taito where to find Petteri's NCO\\
\colorbox{Grey}{\texttt{>>} export PATH=\$PATH:/homeappl/home/puotila/bin/ncks} \\
4. Then launch the code with your own configuration file e.g. ORCA1\_L75\_YG\_taito and your experiment 
\insertcode{"../barakuda_script/taito_barakuda.sh"}{} 
\subsection{Use Opendrift}
OpenDrift is a software for modeling the trajectories and fate of objects or substances drifting in the ocean, or even in the atmosphere.\\
You can get very detailed instructions here on how to build and use it:\\
\url{https://github.com/opendrift/opendrift/wiki}\\
There are several things to do before building it on Taito.\\
\# Note that OpenDrift (as well as some of the requirements) does not support Python3, so that I use python-env/intelpython2.7-2018.2\\
\colorbox{Grey}{\texttt{>>}module load hdf5-par netcdf4 python-env/intelpython2.7-2018.2 ffmpeg imagemagick}\\
\# conda asked me to clone a user root even though I tried to build a local module site package.\\
\colorbox{Grey}{\texttt{>>}conda create -n my\_root --clone="/appl/opt/python/intelpython27-2018.2/intelpython2"}\\
\# go to my root\\
\colorbox{Grey}{\texttt{>>}source activate my\_root}\\
\# now install the modules.\\ 
\colorbox{Grey}{\texttt{>>}conda install --yes --use-local hdf4 numpy scipy matplotlib basemap netcdf4 configobj pillow gdal}\\
\colorbox{Grey}{pyproj ffmpeg}\\
\# then export the path and build.\\
\colorbox{Grey}{\texttt{>>}export PATH=$PATH:$OPENDRIFT\_FOLDER/opendrift/scripts/}\\
\colorbox{Grey}{\texttt{>>}python setup.py develop --user}\\
\# then test.\\
\colorbox{Grey}{\texttt{>>}testall}\\
\end{document}
